\documentclass[10pt, fleqn]{amsart}   

\usepackage[T1]{fontenc} 
%\usepackage[latin1]{inputenc}
%\usepackage[english,francais]{babel}
\usepackage[retainorgcmds]{IEEEtrantools}   \usepackage{graphicx}
\usepackage{amssymb}
\usepackage{pstricks}
\usepackage{pst-node}
\usepackage{pst-plot}
\usepackage{cancel}
\usepackage{empheq}
\usepackage{mathtools}
\usepackage{hyperref}
\usepackage{clrscode3e}
\usepackage{pst-grad,multido}
\usepackage{graphicx,url,etoolbox}
\usepackage{pifont}
\usepackage{subfig}
\usepackage{siunitx}
\usepackage{mathtools}
\usepackage{filecontents}
\usepackage{amsthm}
\usepackage{yhmath}
\usepackage{url}
\usepackage{nicematrix}

%\usepgfplotslibrary{external} 

%\tikzexternalize

\usepackage{pstricks-add}
\let\clipbox\relax
\usepackage{pgfplots}
\usepackage{tikz}
\usetikzlibrary{quantikz}

\makeatletter
\tikzset{
    dot diameter/.store in=\dot@diameter,
    dot diameter=0.5pt,
    dot spacing/.store in=\dot@spacing,
    dot spacing=10pt,
    dots/.style={
        line width=\dot@diameter,
        line cap=round,
        dash pattern=on 0pt off \dot@spacing
    }
}
\makeatother


\usepackage{pifont}

\setlength{\textwidth}{\paperwidth}
\addtolength{\textwidth}{-2in}
\calclayout

\DeclarePairedDelimiter\ceil{\lceil}{\rceil}
\DeclarePairedDelimiter\floor{\lfloor}{\rfloor}

\patchcmd{\section}{\scshape}{\bfseries}{}{}
\makeatletter
\renewcommand{\@secnumfont}{\bfseries}
\makeatother

\newcommand{\ud}{\,\mathrm{d}}
\newcommand{\iu}{{i\mkern1mu}}
\DeclarePairedDelimiter{\abs}{\lvert}{\rvert}

\DeclareMathOperator{\spn}{span}

\usepackage{stmaryrd}
\usepackage{braket}
\usepackage{lipsum}


\usepackage{auto-pst-pdf}
\usepackage{fontawesome5}


\DeclareGraphicsRule{.tif}{png}{.png}{`convert #1 `dirname #1`/`basename #1 .tif`.png}

\theoremstyle{definition}
\newtheorem*{Lemma}{Lemma}
\theoremstyle{definition}
\newtheorem*{Theorem}{Theorem}
\theoremstyle{definition}
\newtheorem*{Definition}{Definition}

\addtolength{\voffset}{-3cm}
\addtolength{\textheight}{4cm}

\makeatletter
\patchcmd{\@settitle}{center}{flushleft}{}{}
\patchcmd{\@settitle}{center}{flushleft}{}{}
\DeclareMathOperator{\Tr}{Tr}
%\patchcmd{\@setauthors}{\centering}{\raggedright}{}{}
\patchcmd{\abstract}{3pc}{0pt}{}{} % remove indentation
\newcommand{\eq}[1]{\begin{IEEEeqnarray*}{rCl}#1\end{IEEEeqnarray*}}
\makeatother

%\usepackage{etoolbox}
% \patchcmd{<cmd>}{<search>}{<replace>}{<success>}{<failure>}
\patchcmd{\section}{\centering}{}{}{}

\title{%
		Quantum Computation and Quantum Information: \\
		Quantum Noise and Quantum Operations}

\begin{document}
\maketitle
\author{Pierre-Paul TACHER}

This document is published under the \href{https://creativecommons.org/licenses/by-nc-sa/4.0/}{Creative Commons Attribution-NonCommercial-ShareAlike 4.0 International license}. \faCreativeCommons\ \faCreativeCommonsBy\ \faCreativeCommonsNc\ \faCreativeCommonsSa

\addtocounter{section}{19}

\section{Circuit model for amplitude damping}

We want to prove that the following circuit models the amplitude damping operation

\begin{center}
\begin{quantikz}
\lstick{$\alpha\ket{0}+\beta\ket{1}$}    &  \ctrl{1}           &  \targ{}  &   \qw   & \qw \\
\lstick{$\ket{0}$} & \gate{R_y(\theta)}  & \ctrl{-1} &  \meter{} & \qw
\end{quantikz}\end{center}

Recall that
\begin{IEEEeqnarray*}{rCl}
R_y(\theta) &= & e^{-\iu \frac{\theta}{2}Y} \\
& = &\begin{bmatrix}
					\cos(\frac{\theta}{2}) & -\sin(\frac{\theta}{2})  \\[1em]
					\sin(\frac{\theta}{2}) &\cos(\frac{\theta}{2})
\end{bmatrix}
\end{IEEEeqnarray*}

Initially the two-qubit state is
\begin{IEEEeqnarray*}{rCl}
 (\alpha\ket{0}+\beta\ket{1})\ket{0}&= & \alpha\ket{00}+\beta\ket{10}  \\
\end{IEEEeqnarray*}

After the controlled $R_y$ gate it becomes
\begin{IEEEeqnarray*}{rCl}
	\alpha\ket{00}+\beta\ket{1}R_y(\theta)\ket{0} 	& =&  \alpha\ket{00}+\beta\ket{1}(\cos(\frac{\theta}{2})\ket{0}+\sin(\frac{\theta}{2})\ket{1})  \\
	& =&  \alpha\ket{00}+\beta(\cos(\frac{\theta}{2})\ket{10}+\sin(\frac{\theta}{2})\ket{11})  \\
\end{IEEEeqnarray*}

After the controlled not gate,
\begin{IEEEeqnarray*}{rCl}
 & & \alpha\ket{00}+\beta(\cos(\frac{\theta}{2})\ket{10}+\sin(\frac{\theta}{2})\ket{01})  \\
\end{IEEEeqnarray*}

This is the effect of amplitude damping, with probability of 1 be switched to 0, or one photon being lost to environment, being $\gamma=\sin^2(\frac{\theta}{2})$.

\section{Amplitude damping of a harmonic oscillator}

The principal system, a harmonic oscillator, interacts with an environment, modeled as another harmonic oscillator, through the Hamiltonian:
\begin{IEEEeqnarray*}{rCl}
H & =& \chi(a^\dagger b + b^\dagger a) \\
\end{IEEEeqnarray*}
where $a^\dagger,a$ and  $b^\dagger,b$ are the creation, annihilation operators for the principal and environment oscillators, respectively.

The time evolution of the coupled system is governed by the unitary operator:
\begin{IEEEeqnarray*}{rCl}
U &= &e^{-\iu H \Delta t}  \\
\end{IEEEeqnarray*}

\subsection{Operation elements} 

We recall some results for the harmonic oscillator:
\begin{IEEEeqnarray*}{rCl}
\forall n \in\mathbb{N},\quad a^\dagger\ket{n} &= &\sqrt{n+1}\ket{n+1}  \\
\end{IEEEeqnarray*}
and similarly in the environment space
\begin{IEEEeqnarray*}{rCl}
	\forall n \in\mathbb{N},\quad b^\dagger\ket{n}_b &= &\sqrt{n+1}\ket{n+1}_b  \\
\end{IEEEeqnarray*}
Here we use the subscript $b$ to differentiate the eigenvectors of the Hermitian operator $bb^\dagger$ which live in the environment space from the eigenvectors of $aa^\dagger$ in the principal space:
\begin{IEEEeqnarray*}{rCl}
\forall n\in\mathbb{N},\quad bb^\dagger\ket{n}_b &= &(n+1)\ket{n}_b  \\
\forall n\in\mathbb{N},\quad aa^\dagger\ket{n} &= &(n+1)\ket{n}  \\
\end{IEEEeqnarray*}
Each set of vectors constitute an orthonormal basis:
\begin{IEEEeqnarray*}{rCl}
\forall (n,m) \in \mathbb{N}^2,\quad \braket{n|m} &= &  \left\{ \,
\begin{IEEEeqnarraybox}[][c]{l?s}
	\IEEEstrut
	0 & if $n \neq m$, \\
	1 & if $n = m$.
	\IEEEstrut
\end{IEEEeqnarraybox}
\right.\\
&= & \delta_{nm} \\
\end{IEEEeqnarray*}

We also have
\begin{IEEEeqnarray*}{rCl}
aa^\dagger -a^\dagger a&= &[a,a^\dagger]  \\
&= &1 \\
bb^\dagger -b^\dagger b&= &[b,b^\dagger]  \\
&= &1 \\
\end{IEEEeqnarray*}
where 1 stands for the identity operator.

Each of the operators $a,a^\dagger$ commutes with each of the operators $b,b^\dagger$ since they act on different spaces
\begin{IEEEeqnarray*}{rCl}
0 &= &[a^\dagger,b^\dagger]  \\
&= &[a,b^\dagger]  \\
&= &[a^\dagger,b]  \\
&= &[a,b]  \\
\end{IEEEeqnarray*}

The Baker-Campbell-Hausdorff formula states that, for any operators $A,G$ such that $e^G$ exists,
\begin{IEEEeqnarray*}{rCl}
e^{\lambda G}Ae^{-\lambda G} &= &\sum_{n=0}^{+\infty} \frac{\lambda^n}{n!}C_n  \\
\end{IEEEeqnarray*}
where the operators $C_n$ are defined recursively by
\begin{IEEEeqnarray*}{rCl}
C_0 & =& A \\
C_1 & = & [G,A] \\
\forall n\in\mathbb{N},\quad C_{n+1} & = & [G,C_n]
\end{IEEEeqnarray*}

Let’s compute a simplified expression for the operator $Ua^\dagger U^\dagger$ acting on the product space:
\begin{IEEEeqnarray*}{rCl}
	Ua^\dagger U^\dagger&= & e^{-\iu H \Delta t} a^\dagger e^{\iu H \Delta t}  \\\IEEEyesnumber 
	& = & \sum_{n=0}^{+\infty} \frac{(-\iu \Delta t)^n}{n!}C_n \label{eq:1}
\end{IEEEeqnarray*}
The first commutators $C_n$ are
\begin{IEEEeqnarray*}{rCl}
C_0 &= &a^\dagger  \\
C_1 & = & [H,a^\dagger] \\
& = & [\chi b^\dagger a,a^\dagger]\\
& = & \chi b^\dagger[ a,a^\dagger]\\
& = & \chi b^\dagger\\
C_2 & = & [H,C_1] \\
& = & [\chi a^\dagger b,\chi b^\dagger ]\\
& = & \chi^2 a^\dagger[ b,b^\dagger]\\
& = & \chi^2 a^\dagger\\
\end{IEEEeqnarray*}
from which it follows that
\begin{IEEEeqnarray*}{rCl}
\forall n\in\mathbb{N},\quad C_{2n} &= \chi^{2n}a^\dagger&  \\
C_{2n+1} &= \chi^{2n+1}b^\dagger&  \\
\end{IEEEeqnarray*}

We now rewrite equation~\ref{eq:1}
\begin{IEEEeqnarray*}{rCl}
	Ua^\dagger U^\dagger & = & \sum_{n=0}^{+\infty} \frac{(-\iu \Delta t)^n}{n!}C_n \\
	& = & \sum_{n=0}^{+\infty} \frac{(-\iu \Delta t)^{2n}}{(2n)!}C_{2n}+\sum_{n=0}^{+\infty} \frac{(-\iu \Delta t)^{2n+1}}{(2n+1)!}C_{2n+1} \\
	& = & a^\dagger\sum_{n=0}^{+\infty} \frac{(-\iu \chi\Delta t)^{2n}}{(2n)!}+b^\dagger\sum_{n=0}^{+\infty} \frac{(-\iu \chi \Delta t)^{2n+1}}{(2n+1)!} \\
	& = & a^\dagger\sum_{n=0}^{+\infty} (-1)^n\frac{ (\chi\Delta t)^{2n}}{(2n)!}-\iu b^\dagger\sum_{n=0}^{+\infty} (-1)^n\frac{( \chi \Delta t)^{2n+1}}{(2n+1)!} \\
	& = & \cos(\chi \Delta t)a^\dagger -i\sin(\chi\Delta t)b^\dagger
\end{IEEEeqnarray*}

Let us now compute the effect of $U$ on $\ket{0}\ket{0}_b=\ket{00}$:
\begin{IEEEeqnarray*}{rCl}
U\ket{00} &= &e^{-\iu H \Delta t} \ket{00} \\
& = &\sum_{n=0}^{+\infty} \frac{(-\iu H\Delta t)^n}{n!} \ket{00}
\end{IEEEeqnarray*}
Since $a\ket{0}=0$ and $b\ket{0}_b=0$, we have 
\begin{IEEEeqnarray*}{rCl}
H\ket{00} &= & 0 \\
\end{IEEEeqnarray*}
and 
\begin{IEEEeqnarray*}{rCl}
\forall n\in\mathbb{N}^*,\quad,H^n \ket{00}& =&0  \\
\end{IEEEeqnarray*}
from which it follows there is only one non nul term in the previous sum and
\begin{IEEEeqnarray*}{rCl}
U\ket{00} &= &\ket{00}  \\
\end{IEEEeqnarray*}

Let us compute the effect of $U$ on $\ket{1}\ket{0}_b=\ket{10}$:
\begin{IEEEeqnarray*}{rCl}
U\ket{10} &= &Ua^\dagger\ket{00}  \\
&= &Ua^\dagger \underbrace{U^\dagger U}_{=1}\ket{00}  \\
&= &Ua^\dagger U^\dagger\ket{00}  \\
&= &( \cos(\chi \Delta t)a^\dagger -i\sin(\chi\Delta t)b^\dagger)\ket{00}  \\
&= & \cos(\chi \Delta t) \ket{10} -i\sin(\chi\Delta t) \ket{01}  \\
&= & \cos(\chi \Delta t )\ket{1}\ket{0}_b -i\sin(\chi\Delta t )\ket{0}\ket{1}_b  \\
\end{IEEEeqnarray*}

Similarly,
\begin{IEEEeqnarray*}{rCl}
\sqrt{n!}U\ket{n}\ket{0}_b &= & \sqrt{n!}U\ket{n0} \\
&= & U(a^\dagger)^n\ket{00} \\
&= & U(a^\dagger)^nU^\dagger U\ket{00} \\
&= & (Ua^\dagger U^\dagger )^n\ket{00} \\
&= & (\cos(\chi \Delta t)a^\dagger -i\sin(\chi\Delta t)b^\dagger )^n\ket{00} \\
\end{IEEEeqnarray*}
Since $[a^\dagger,b^\dagger]=0$,
\begin{IEEEeqnarray*}{rCl}
	\sqrt{n!}U\ket{n}\ket{0}_b&= &\left(\sum_{k=0}^n\binom{n}{k}\cos^{n-k}(\chi\Delta t)(-\iu)^k\sin^k(\chi\Delta t)(a^\dagger)^{n-k}(b^\dagger)^{k}\right) \ket{00} \\
	&= &\sum_{k=0}^n\binom{n}{k}\cos^{n-k}(\chi\Delta t)(-\iu)^k\sin^k(\chi\Delta t) \sqrt{(n-k)!}\sqrt{k!}\ket{n-k}\ket{k}_b \\
\end{IEEEeqnarray*}
so that
\begin{IEEEeqnarray*}{rCl}
U\ket{n0} &= & \sum_{k=0}^n\binom{n}{k}\sqrt{\frac{(n-k)!k!}{n!}}\cos^{n-k}(\chi\Delta t)(-\iu)^k\sin^k(\chi\Delta t) \ket{n-k}\ket{k}_b \\
&= & \sum_{k=0}^n\sqrt{\binom{n}{k}}\cos^{n-k}(\chi\Delta t)(-\iu)^k\sin^k(\chi\Delta t) \ket{n-k}\ket{k}_b \\
\end{IEEEeqnarray*}
We can think of the number
\begin{IEEEeqnarray*}{rCl}
 & &\binom{n}{k}\cos^{2(n-k)}(\chi\Delta t)\sin^{2k}(\chi\Delta t)   \\
\end{IEEEeqnarray*}
as the probability of losing $k$ quanta of energy to the environment.

Let $E_m=\bra{m}_b U \ket{0}_b,\quad m\in\mathbb{N}$ the operation elements of $U$. They are operators acting on the principal space. We can compute the action of $E_m$ on $\ket{n}$ (i.e. compute the nth column of the matrix of $E_m$) from the previous formula:
\begin{IEEEeqnarray*}{rCl}
E_m\ket{n} & =& (\bra{m}_b U \ket{0}_b)\ket{n} \\
& =& \bra{m}_b (U \ket{n}\ket{0}_b) \\
& =& \bra{m}_b U \ket{n0}\\
\end{IEEEeqnarray*}
First it is clear that if $n<m$, $E_m\ket{n}=0$. Then if $n\geqslant m$,
\begin{IEEEeqnarray*}{rCl}
	E_m\ket{n} & =&\bra{m}_b  \sum_{k=0}^n\sqrt{\binom{n}{k}}\cos^{n-k}(\chi\Delta t)(-\iu)^k\sin^k(\chi\Delta t) \ket{n-k}\ket{k}_b \\
	& =& \sum_{k=0}^n\sqrt{\binom{n}{k}}\cos^{n-k}(\chi\Delta t)(-\iu)^k\sin^k(\chi\Delta t) \ket{n-k}\underbrace{\braket{m|k}_b}_{=\delta_{mk}} \\
	& =& (-\iu)^m\sin^m(\chi\Delta t) \sqrt{\binom{n}{m}}\cos^{n-m}(\chi\Delta t)\ket{n-m} \\
\end{IEEEeqnarray*}
This shows that the matrix of $E_m$ has non nul elements only on the mth superior diagonal. $E_m$ corresponds to the physical process of losing $m$ quanta of energy to the environment. 

We can also reconstruct the full formula for $E_m$ using braket calculus:
\begin{IEEEeqnarray*}{rCl}
E_m & =& E_m\underbrace{\sum_{n=0}^{+\infty}\ket{n}\bra{n}}_{=1} \\
& =& \sum_{n=0}^{+\infty}E_m\ket{n}\bra{n} \\
& =& \sum_{n=m}^{+\infty}E_m\ket{n}\bra{n} \\
& =&(-\iu)^m\sin^m(\chi\Delta t)  \sum_{n=m}^{+\infty}\sqrt{\binom{n}{m}}\cos^{n-m}(\chi\Delta t)\ket{n-m}\bra{n} \\
\end{IEEEeqnarray*}
Note that the sole effect of factor $(-\iu)^m$  is to add a global phase so it may as well be omitted.

\setcounter{MaxMatrixCols}{20}
\NiceMatrixOptions{cell-space-limits = 1pt}
\pgfdeclarelayer{bg}
\pgfsetlayers{bg,main}
\begin{IEEEeqnarray*}{rCl}
E_m %&=&P\begin{bmatrix}G(\lambda) &0 &\ldots & 0& 0 & \ldots&\ldots&0\\
   %  0 & G(\lambda)  & \ldots & 0&\vdots & &&\vdots\\
   %\vdots & \vdots & \ddots & \vdots&\vdots & &&\vdots\\
   %0 & 0 & \ldots &   G(\lambda) &0&\ldots &\ldots &0\\
   %0 & \dots & \ldots &  0&F(\lambda_1) &0 & \dots&0 \\
   %\vdots &  &  & \vdots &0 & F(\lambda_2)& &\vdots\\
   %\vdots &  &  & \vdots & \vdots& &\ddots &\vdots\\
   %0 & \ldots & \ldots &  0&0 & \dots& \ldots&F(\lambda_{n-d})\\
   %\end{bmatrix} \\
   %&=&P\begin{bNiceMatrix}[nullify-dots]
   %  G(\lambda) &\text{\kern -0.71667em}0\hphantom{\text{a}} &\Ldots & & 0        & 0            & \Ldots& & & 0      \\
   %  0          & \text{\kern -0.71667em}G(\lambda)          &       & &          &\Vdots        &       & & & \Vdots \\
   %\Vdots       & \Ddots                                     & \Ddots& &\Vdots    &\Vdots        &       & & &\Vdots  \\
   %\Vdots       &                                            & \Ddots& &\Vdots    &\Vdots        &       & & &\Vdots  \\
   %0            & \Ldots                                     &       &0&G(\lambda)& 0            &\Ldots & & &0        \\
   %0            &                                            & \Ldots& &0         &F(\lambda_1)  &\text{\kern -0.71667em}0\hphantom{\text{a}}  & \Ldots&&0 \\
   %\Vdots       &                                            &       & &\Vdots    &0             &\text{\kern -0.71667em} F(\lambda_2)         &       &&\Vdots\\
   %\Vdots       &                                            &       & &\Vdots    &\Vdots        &\Ddots                                       &\Ddots &&\Vdots\\
   %\Vdots       &                                            &       & &\Vdots    &\Vdots        &                                             &       &&\Vdots\\
   %0            & \Ldots                                     & \Ldots& &  0       &0             & \Ldots                                      &       &0&F(\lambda_{n-d})\\\end{bNiceMatrix}\\
   %\fi
   & = &\sin^m(\chi\Delta t)\begin{bNiceMatrix}[nullify-dots,first-row,last-col]
	&                                            &       & &    &        &       & m& &  &&n&&&&\\
	 & & & &         &             & &1 &  &       &\Block[c]{3-3}<\LARGE>{\boldsymbol{0}} &\\
	           & \Block[c]{14-10}<\LARGE>{\boldsymbol{0}}          &       & &          &        &       & &\text{\kern -0.71667em}\sqrt{\binom{m+1}{m}}\cos(\chi\Delta t) &  &&&&&\\
          &                                      &       & &    &        &       & & &\text{\kern -0.91667em}\sqrt{\binom{m+2}{m}} \cos^2(\chi\Delta t) &&&&&\\
          &                                            &       & &    &        &       & & & && &&&\\
               &                                    &       &&&             & & & &        &&\text{\kern -0.91667em}\sqrt{\binom{n}{m}}\cos^{n-m}(\chi\Delta t)&&&&n-m\\
               &                                            & & &         &  &  &       && &&&&&\\
         &                                            &       & &    &             &                 &       &&&&&&\text{\kern -0.91667em}\vphantom{\sqrt{\binom{n}{m}}\cos^{n-m}(\chi\Delta t)}\NotEmpty&\\
          &                                            &       & &    &        &                                                &       &&&&&&&\NotEmpty\\
          &                                            &       & &    &        &                                                      &       &&&&&&&\\
		  &                                            &       & &    &        &       & & &  &&&&&\\
		  &                                            &       & &    &        &       & & &  &&&&&\\
		  &                                            &       & &    &        &       & & &  &&&&&\\
		  &                                            &       & &    &        &       & & &  &&&&&\\
		  &                                            &       & &    &        &       & & &  &&&&&\\
               &                                      & & &         &             &             &&                                   &       &&&&&
   \CodeAfter
   \begin{tikzpicture}[]
	
	%\draw (1,1) -- (2,1)--(0,0);
   %\draw (2-2) circle (2mm);
   %\draw (5-5) circle (2mm);
   %\draw (2-2) -- (5-5);
   %\draw (-5,-5) -- (5,5);
   \draw[dot diameter=1pt, dot spacing=5pt, dots] (-7,1.97) -- (-5.1,1.27);
   \draw[dot diameter=1pt, dot spacing=5pt, dots] (-4,0.87) -- (-1.2,-0.16);
   
   %\draw[dot diameter=1pt, dot spacing=5pt, dots] (2-2) -- (5-5.north);
   %\draw[dot diameter=1pt, dot spacing=5pt, dots] (3-10.south) -- (5-12.west);
   %\draw[dot diameter=1pt, dot spacing=5pt, dots] (5-12.south) -- (7-14.west);
   %\draw[dot diameter=1pt, dot spacing=5pt, dots] (1-2.south) -- (1-5.south west);
   %\draw[dot diameter=1pt, dot spacing=5pt, dots] (6-7.south) -- (6-10.south west);
   %\line{2-2}{5-5}.north east;
   %\line{2-2}{1-5}.north;

   \begin{pgfonlayer}{bg}
	%\draw (1-8) -- (8-15); 
   \end{pgfonlayer}
   \end{tikzpicture}\end{bNiceMatrix}
\end{IEEEeqnarray*}

\subsection{Trace-preserving property}

Matrix calculus or braket calculus show that the matrices $E_m^\dagger E_m$ are diagonals, with the first $m$ elements are 0:
\begin{IEEEeqnarray*}{rCl}
	E_m^\dagger E_m & =&  \sin^{2m}(\chi\Delta t)\left(  \sum_{n=m}^{+\infty}\sqrt{\binom{n}{m}}\cos^{n-m}(\chi\Delta t)\ket{n}\bra{n-m}\right)\left(  \sum_{l=m}^{+\infty}\sqrt{\binom{l}{m}}\cos^{l-m}(\chi\Delta t)\ket{l-m}\bra{l}\right)\\
	& =&  \sin^{2m}(\chi\Delta t) \sum_{n=m}^{+\infty} \sum_{l=m}^{+\infty}\sqrt{\binom{n}{m}}\sqrt{\binom{l}{m}}\cos^{n-m}(\chi\Delta t)\cos^{l-m}(\chi\Delta t)\ket{n}\underbrace{\braket{n-m|l-m}}_{=\delta_{nl}}\bra{l}\\
	& =&  \sin^{2m}(\chi\Delta t) \sum_{n=m}^{+\infty} \binom{n}{m}\cos^{2(n-m)}(\chi\Delta t)\ket{n}\bra{n}
\end{IEEEeqnarray*}

\setcounter{MaxMatrixCols}{20}
\NiceMatrixOptions{cell-space-limits = 2pt}
\pgfdeclarelayer{bg}
\pgfsetlayers{bg,main}
\begin{IEEEeqnarray*}{rCl}
E_m^\dagger E_m %&=&P\begin{bmatrix}G(\lambda) &0 &\ldots & 0& 0 & \ldots&\ldots&0\\
   %  0 & G(\lambda)  & \ldots & 0&\vdots & &&\vdots\\
   %\vdots & \vdots & \ddots & \vdots&\vdots & &&\vdots\\
   %0 & 0 & \ldots &   G(\lambda) &0&\ldots &\ldots &0\\
   %0 & \dots & \ldots &  0&F(\lambda_1) &0 & \dots&0 \\
   %\vdots &  &  & \vdots &0 & F(\lambda_2)& &\vdots\\
   %\vdots &  &  & \vdots & \vdots& &\ddots &\vdots\\
   %0 & \ldots & \ldots &  0&0 & \dots& \ldots&F(\lambda_{n-d})\\
   %\end{bmatrix} \\
   %&=&P\begin{bNiceMatrix}[nullify-dots]
   %  G(\lambda) &\text{\kern -0.71667em}0\hphantom{\text{a}} &\Ldots & & 0        & 0            & \Ldots& & & 0      \\
   %  0          & \text{\kern -0.71667em}G(\lambda)          &       & &          &\Vdots        &       & & & \Vdots \\
   %\Vdots       & \Ddots                                     & \Ddots& &\Vdots    &\Vdots        &       & & &\Vdots  \\
   %\Vdots       &                                            & \Ddots& &\Vdots    &\Vdots        &       & & &\Vdots  \\
   %0            & \Ldots                                     &       &0&G(\lambda)& 0            &\Ldots & & &0        \\
   %0            &                                            & \Ldots& &0         &F(\lambda_1)  &\text{\kern -0.71667em}0\hphantom{\text{a}}  & \Ldots&&0 \\
   %\Vdots       &                                            &       & &\Vdots    &0             &\text{\kern -0.71667em} F(\lambda_2)         &       &&\Vdots\\
   %\Vdots       &                                            &       & &\Vdots    &\Vdots        &\Ddots                                       &\Ddots &&\Vdots\\
   %\Vdots       &                                            &       & &\Vdots    &\Vdots        &                                             &       &&\Vdots\\
   %0            & \Ldots                                     & \Ldots& &  0       &0             & \Ldots                                      &       &0&F(\lambda_{n-d})\\\end{bNiceMatrix}\\
   %\fi
   & = &\sin^{2m}(\chi\Delta t)\begin{bNiceMatrix}[nullify-dots,first-row,last-col]
	&                                            &       & &  m  &        &       & & n&  &&&&&&\\
	\NotEmpty\Block[fill=lightgray,rounded-corners]{4-4}<\LARGE>{\boldsymbol{0}} & & & &         &             & & &  &       & &\\
	           & \hphantom{\cos(\Delta )}         &       & &          &        &       & & & \Block[c]{3-3}<\LARGE>{\boldsymbol{0}} &&&&&\\
          &                                      &     \hphantom{\cos(\Delta )}   & &    &        &       & & & &&&&&\\
          &                                            &       &\NotEmpty&    &        &       & & & && &&&\\
               &                                    &       &&1&             & & & &        &&&&&&m\\
               &                                            & & &         & \text{\kern -0.71667em}\binom{m+1}{m}\cos^2(\chi\Delta t) &  &       && &&&&&\\
         &                                            &       & &    &             &    \text{\kern -3.9667em}\binom{m+1}{m}\cos^{4}(\chi\Delta t)              &       &&&&&&&\\
          &                                            &       & &    &        &                                                &       &&&&&&&\NotEmpty\\
          &                                            &       & &    &        &                                                      &       &&&&&&&\\
		  &                                            &       & &    &        &       & & & \text{\kern -3.9667em}\binom{n}{m}\cos^{2(n-m)}(\chi\Delta t)  &&&&&&n\\
		  &                                            &       & &    &        &       & & &  &&&&&\\
		  &    \Block[c]{3-3}<\LARGE>{\boldsymbol{0}}                                        &       & &    &        &       & & &  &&&&&\\
		  &                                            &       & &    &        &       & & &  &&&&&\\
		  &                                            &       & &    &        &       & & &  &&&&&\\
               &                                      & & &         &             &             &&                                   &       &&&&&
   \CodeAfter
   \begin{tikzpicture}[]	
	%\draw (1,1) -- (2,1)--(0,0);
   %\draw (2-2) circle (2mm);
   %\draw (5-5) circle (2mm);
   %\draw (2-2) -- (5-5);
   %\draw (-5,-5) -- (5,5);
   \draw[dot diameter=1pt, dot spacing=5pt, dots] (-6.5,0.34) -- (-4.5,-0.69);
   \draw[dot diameter=1pt, dot spacing=5pt, dots] (-4,-0.95) -- (-0.75,-2.6);
   %\draw[dot diameter=1pt, dot spacing=5pt, dots] (-4,0.87) -- (-1.2,-0.16);
   
   %\draw[dot diameter=1pt, dot spacing=5pt, dots] (1-1) -- (4-4);
   %\draw[dot diameter=1pt, dot spacing=5pt, dots] (3-10.south) -- (5-12.west);
   %\draw[dot diameter=1pt, dot spacing=5pt, dots] (5-12.south) -- (7-14.west);
   %\draw[dot diameter=1pt, dot spacing=5pt, dots] (1-2.south) -- (1-5.south west);
   %\draw[dot diameter=1pt, dot spacing=5pt, dots] (6-7.south) -- (6-10.south west);
   %\line{2-2}{5-5}.north east;
   %\line{2-2}{1-5}.north;

   \begin{pgfonlayer}{bg}
	%\draw (1-8) -- (8-15); 
   \end{pgfonlayer}
   \end{tikzpicture}\end{bNiceMatrix}
\end{IEEEeqnarray*}

It follows that the operator $\sum_{m=0}^{+\infty}E_m^\dagger E_m$ is also diagonal, and diagonal elements are
\begin{IEEEeqnarray*}{rCl}
\braket{n|\sum_{m=0}^{+\infty}E_m^\dagger E_m|n} &= &\sum_{m=0}^{+\infty} \braket{n|E_m^\dagger E_m|n} \\
&= &\sum_{m=0}^{n} \braket{n|E_m^\dagger E_m|n} \\
&= &\sum_{m=0}^{n} \binom{n}{m}\sin^{2m}(\chi\Delta t)\cos^{2(n-m)}(\chi \Delta t) \\
&= &(\sin^{2}(\chi\Delta t)+\cos^{2}(\chi \Delta t))^n \\
&= &1 \\
\end{IEEEeqnarray*}
i.e.\ $\sum_{m=0}^{+\infty}E_m^\dagger E_m=1$ and the quantum operation is trace-preserving.

\section{Amplitude damping of a single qubit density matrix}

Let
\begin{IEEEeqnarray*}{rCl}
\rho &= & \begin{bmatrix} a & b \\
	b^* & c 
\end{bmatrix}
\end{IEEEeqnarray*}

The amplitude damping operation is defined by
\begin{IEEEeqnarray*}{rCl}
\varepsilon_{AD}(\rho) &= & E_0\rho E_0^\dagger + E_1\rho E_1^\dagger  \\
\end{IEEEeqnarray*}
where
\begin{IEEEeqnarray*}{rCl}
E_0	 &= &\begin{bmatrix}
	1 & 0   \\
	0 & \sqrt{1-\gamma}
\end{bmatrix} \\\IEEEyesnumber
E_1 & = & \begin{bmatrix}
	0 & \sqrt{\gamma}   \\
	0 & 0
\end{bmatrix} \label{eq:3}\\
\end{IEEEeqnarray*}

Straightforward matrix calculus show that
\begin{IEEEeqnarray*}{rCl}
	E_0\rho E_0^\dagger  & = &  \begin{bmatrix}
		a & b\sqrt{1-\gamma}   \\
		b^*\sqrt{1-\gamma} & c(1-\gamma)
	\end{bmatrix} \\
\end{IEEEeqnarray*}
and
\begin{IEEEeqnarray*}{rCl}
	E_1\rho E_1^\dagger  & = &  \begin{bmatrix}
		c\gamma & 0   \\
		0 & 0
	\end{bmatrix} \\
	& = &  \begin{bmatrix}
		(1-a)\gamma & 0   \\
		0 & 0
	\end{bmatrix} \\
\end{IEEEeqnarray*}
 because $1=\Tr \rho=a+c$.

Thus we have
\begin{IEEEeqnarray*}{rCl}\IEEEyesnumber
\varepsilon_{AD}(\rho) &= &  \begin{bmatrix}
	a + (1-a)\gamma & b\sqrt{1-\gamma}   \\
	b^*\sqrt{1-\gamma} & c(1-\gamma) \label{eq:5}
\end{bmatrix} \\
&= &  \begin{bmatrix}
	1 - (1-a)(1-\gamma) & b\sqrt{1-\gamma}   \\
	b^*\sqrt{1-\gamma} & c(1-\gamma)
\end{bmatrix} \\
\end{IEEEeqnarray*}

\section{Amplitude damping of dual-rail qubits}

Let 
\begin{IEEEeqnarray*}{rCl}
\ket{\psi} &= & a\ket{01} + b\ket{10} \\
\end{IEEEeqnarray*}

Applying $\varepsilon_{AD}\otimes \varepsilon_{AD}$ to $\rho=\ket{\psi}\bra{\psi}$ is equivalent to applying unitary $B\otimes B$ to $\ket{\psi}$, where $B=e^{\theta(a^\dagger b-ab^\dagger)}$. Let's do this by making explicit the 2 environment qubits initially set to 0, dnoted by subscript $b$:
\begin{IEEEeqnarray*}{rCl}
	\ket{\psi} &= & a\ket{01}\ket{00}_b + b\ket{10}\ket{00}_b \\
\end{IEEEeqnarray*}
\begin{IEEEeqnarray*}{rCl}
	B\otimes B	\ket{\psi} &= & a\ket{0}\ket{0}_b(B\ket{1}\ket{0}_b) + b(B\ket{1}\ket{0}_b)\ket{0}\ket{0}_b \\
	&= & a\ket{0}\ket{0}_b(\cos(\theta)\ket{1}\ket{0}_b+\sin(\theta)\ket{0}\ket{1}_b) + b(\cos(\theta)\ket{1}\ket{0}_b+\sin(\theta)\ket{0}\ket{1}_b)\ket{0}\ket{0}_b \\
	&= & a\cos(\theta)\ket{0}\ket{0}_b\ket{1}\ket{0}_b+a\sin(\theta)\ket{0}\ket{0}_b\ket{0}\ket{1}_b + b\cos(\theta)\ket{1}\ket{0}_b\ket{0}\ket{0}_b+b\sin(\theta)\ket{0}\ket{1}_b\ket{0}\ket{0}_b \\
\end{IEEEeqnarray*}

We reorder the qubits to put the environments qubits at the end since we will trace them out:
\begin{IEEEeqnarray*}{rCl}
	\IEEEyesnumber B\otimes B	\ket{\psi} &= & a\cos(\theta)\ket{01}\ket{00}_b+a\sin(\theta)\ket{00}\ket{01}_b + b\cos(\theta)\ket{10}\ket{00}_b+b\sin(\theta)\ket{00}\ket{10}_b \label{eq:2}\\
	& = & \ket{\varphi}
\end{IEEEeqnarray*}

Now we have to find the dual vector $\bra{\varphi}$ of this state. We can recall the not so trivial following facts related to product space:
Let $\{\ket{a_i}\}$,$\{\ket{b_j}\}$ be basis of two Hilbert spaces $A$ and $B$.

The dual of $\ket{a_ib_j}=\ket{a_i}\otimes\ket{b_j}$ is 
\begin{IEEEeqnarray*}{rCl}
	\bra{a_i}\otimes\bra{b_j} & =&  \bra{a_ib_j} \\
\end{IEEEeqnarray*}
so that
\begin{IEEEeqnarray*}{rCl}
\bra{\varphi} & =&   a^*\cos(\theta)\bra{01}\bra{00}_b+a^*\sin(\theta)\bra{00}\bra{01}_b + b^*\cos(\theta)\bra{10}\bra{00}_b+b^*\sin(\theta)\bra{00}\bra{10}_b\\
\end{IEEEeqnarray*}

We have also
\begin{IEEEeqnarray*}{rCl}
	 \ket{a_kb_l} \bra{a_ib_j} & = & \ket{a_k}\bra{a_i}\otimes \ket{b_l}\bra{b_j} \\
\end{IEEEeqnarray*}

We could then use equation~\ref{eq:2} to compute the density $\ket{\varphi}\bra{\varphi}$, but this would be a messy sum with 16 terms.

Since we will trace out the environment, we recall the partial trace formula:
\begin{IEEEeqnarray*}{rCl}
\Tr_B (\ket{a_k}\bra{a_i}\otimes \ket{b_l}\bra{b_j})  &= & \ket{a_k}\bra{a_i}\Tr(\ket{b_l}\bra{b_j}) \\
&= & \ket{a_k}\bra{a_i}\braket{b_l | b_j} \\
\end{IEEEeqnarray*}

Since $\{\ket{00}_b,\ket{01}_b,\ket{10}_b,\ket{11}_b\}$ is an orthonormal basis, there are only 6 out of 16 terms left after the partial trace operation:
\begin{IEEEeqnarray*}{rCl}
\Tr_b(\ket{\varphi}\bra{\varphi}) & =& \abs{a}^2\cos^2(\theta)\ket{01}\bra{01} +  ab^*\cos^2(\theta)\ket{01}\bra{10} + \abs{a}^2\sin^2(\theta)\ket{00}\bra{00} \\
&&+\:  \abs{b}^2\cos^2(\theta)\ket{10}\bra{10} + ba^*\cos^2(\theta)\ket{10}\bra{01} +  \abs{b}^2\sin^2(\theta)\ket{00}\bra{00} \\
& =& \abs{a}^2(1-\gamma)\ket{01}\bra{01} +  ab^*(1-\gamma)\ket{01}\bra{10} + \abs{a}^2\gamma\ket{00}\bra{00} \\
&&+\:  \abs{b}^2(1-\gamma)\ket{10}\bra{10} + ba^*(1-\gamma)\ket{10}\bra{01} +  \abs{b}^2\gamma\ket{00}\bra{00} \\
& = & (\underbrace{\abs{a}^2+\abs{b}^2}_{=1})\gamma\ket{00}\bra{00}+(1-\gamma)\left(\abs{a}^2\ket{01}\bra{01} +  ab^*\ket{01}\bra{10} +   \abs{b}^2\ket{10}\bra{10} + ba^*\ket{10}\bra{01}\right)\\
& = &\gamma\begin{bmatrix}
	1 & 0 &0&0\\
	0&0&0&0\\
	0&0&0&0\\
	0&0&0&0
\end{bmatrix}+(1-\gamma)\begin{bmatrix}
	0 & 0 &0&0\\
	0& \abs{a}^2&ab^*&0\\
	0&a^*b&\abs{b}^2&0\\
	0&0&0&0
\end{bmatrix}\\
& = &  \gamma\begin{bmatrix}
	1 & 0 &0&0\\
	0&0&0&0\\
	0&0&0&0\\
	0&0&0&0
\end{bmatrix}+(1-\gamma)\rho
\end{IEEEeqnarray*}

It is a mixed state: 
\begin{itemize}
	\renewcommand\labelitemi{--}
	\item with probability $\gamma$, the state is projected to $\ket{00}$, orthogonal to $\ket{\psi}$.
	\item with probability $1-\gamma$, state is unchanged.
\end{itemize}

Since $\ket{00}$ is orthogonal to $\ket{\psi}$, one can detect amplitude damping errors with measurement operators:
\begin{IEEEeqnarray*}{rCl}
M_0 & = & \ket{00}\bra{00}\qquad\text{orthogonal projector on }\spn\{\ket{00}\} \\
M_1 & = & \ket{01}\bra{01}+\ket{10}\bra{10}+\ket{11}\bra{11}\qquad\text{orthogonal projector on }\spn\{\ket{01},\ket{10}\,\ket{11}\} \\
\end{IEEEeqnarray*}

\begin{itemize}
	\renewcommand\labelitemi{--}
	\item If the state decayed to $\ket{00}$, then with probability $1$ the result of the measurement will be $\ket{00}$.
	\item Otherwise, with probability $1$ the result of the measurement will be the original $\ket{\psi}$.
\end{itemize}

It can be easily checked that the quantum operation can be described with 3 operators:
\begin{IEEEeqnarray*}{rCl}
E_0^{dr} &= & \sqrt{1-\gamma}I \\
E_1^{dr} & =& \sqrt{\gamma}\ket{00}\bra{01} \\
& = & \sqrt{\gamma}\begin{bmatrix}
	 0& 1 &0&0\\
	0&0&0&0\\
	0&0&0&0\\
	0&0&0&0
\end{bmatrix} \\
E_2^{dr} & =& \sqrt{\gamma}\ket{00}\bra{10} \\
& = & \sqrt{\gamma}\begin{bmatrix}
	0& 0 &1&0\\
   0&0&0&0\\
   0&0&0&0\\
   0&0&0&0
\end{bmatrix} \\
\end{IEEEeqnarray*}
 It is interesting to see that these operators are the restriction to $\spn\{\ket{01},\ket{10}\}$ of the operators
 \begin{IEEEeqnarray*}{rCl}
 E_0\otimes E_0 & &  \\
 E_0\otimes E_1 & &  \\
 E_1\otimes E_0 & &  \\
 E_1\otimes E_1 & &  \\
 \end{IEEEeqnarray*}
where $E_0,E_1$ are the operators of amplitude damping for single qubit, defined in~\ref{eq:3}.

\section{Spontaneous emission is amplitude damping}
\iffalse
We consider a system formed by a two-level atom and a cavity confined electric field. The Hamiltonian is
\begin{IEEEeqnarray*}{rCl}
H &= &  g(a\sigma_-+a^\dagger\sigma_+)\\
\end{IEEEeqnarray*}
where $g$ is some constant which describes the strength of the interaction, $a^\dagger,a$ are respectively the creation,annihilation operators~\footnote{It seems to me the book mixes up $a^\dagger$ with $a$ in several places.} on the single mode field, and $\sigma_\pm$ are operators acting on the two-level atom, namely:
\begin{IEEEeqnarray*}{rCl}
\sigma_+ & =& \frac{1}{2}(X+\iu Y) \\
& = &\begin{bmatrix}
       0 & 1\\
       0 & 0
    \end{bmatrix}\\
\sigma_- & =& \frac{1}{2}(X-\iu Y) \\
& = &\begin{bmatrix}
    0 & 0\\
    1 & 0
 \end{bmatrix}\\
\end{IEEEeqnarray*}


We recall
\begin{IEEEeqnarray*}{rCl}
\forall n \in\mathbb{N},\quad a^\dagger\ket{n} &= &\sqrt{n+1}\ket{n+1}  \\
a\ket{n+1} &= &\sqrt{n+1}\ket{n}  \\
a\ket{0} & = &0
\end{IEEEeqnarray*}
\fi

From equation (7.77) in the book, the time evolution of the single atom interacting with single photon is governed by unitary
\begin{IEEEeqnarray*}{rCl}
U &= & e^{-\iu \delta t}\ket{00}\bra{00} + (\cos(\Omega t)+\iu \frac{\delta}{\Omega} \sin(\Omega t))\ket{01}\bra{01}\\
&& \qquad +\>  (\cos(\Omega t)-\iu \frac{\delta}{\Omega} \sin(\Omega t))\ket{10}\bra{10} -\iu \frac{g}{\Omega} \sin(\Omega t)(\ket{01}\bra{10}+\ket{10}\bra{01} ) \\
\end{IEEEeqnarray*}

the left label corresponds to the electric field, the right label corresponds to the atom. The derivation of this formula from the Hamiltonian can be found in appendix~\ref{unitary}.

The {\em Rabi frequency} is 
\begin{IEEEeqnarray*}{rCl}
\Omega &= &\sqrt{g^2+\delta^2}  \\
\end{IEEEeqnarray*}

If we set $\delta=0$ and if $g>0$, then $\Omega=g$ and
\begin{IEEEeqnarray*}{rCl}
U & =& \ket{00}\bra{00} + \cos(\Omega t)(\ket{01}\bra{01}+\ket{10}\bra{10} ) \\
&& \qquad -\>   \iu  \sin(\Omega t)(\ket{01}\bra{10}+\ket{10}\bra{01} ) \\
\end{IEEEeqnarray*}

Let us apply $U$ to 
\begin{IEEEeqnarray*}{rCl}
\ket{\psi} & = &\ket{0}(a\ket{0}+b\ket{1}) \\
& = &a\ket{00}+b\ket{01}) \\
\end{IEEEeqnarray*}

We find
\begin{IEEEeqnarray*}{rCl}
 U\ket{\psi} & =& a\ket{00} + b(\cos(\Omega t)\ket{01}-\iu \sin(\Omega t)\ket{10})  \\\IEEEyesnumber
 & = & \ket{\varphi} \label{eq:4}
\end{IEEEeqnarray*}


Now we have to find the dual vector $\bra{\varphi}$ of this state. We can recall the not so trivial following facts related to product space:
Let $\{\ket{a_i}\}$,$\{\ket{b_j}\}$ be basis of two Hilbert spaces $A$ and $B$.

The dual of $\ket{a_ib_j}=\ket{a_i}\otimes\ket{b_j}$ is 
\begin{IEEEeqnarray*}{rCl}
	\bra{a_i}\otimes\bra{b_j} & =&  \bra{a_ib_j} \\
\end{IEEEeqnarray*}
so that
\begin{IEEEeqnarray*}{rCl}
\bra{\varphi} & =&   a\bra{00} + b^*(\cos(\Omega t)\bra{01}+\iu \sin(\Omega t)\bra{10})  \\
\end{IEEEeqnarray*}

We have also
\begin{IEEEeqnarray*}{rCl}
	 \ket{a_kb_l} \bra{a_ib_j} & = & \ket{a_k}\bra{a_i}\otimes \ket{b_l}\bra{b_j} \\
\end{IEEEeqnarray*}

We could then use equation~\ref{eq:4} to compute the density $\ket{\varphi}\bra{\varphi}$, but this would be a ugly sum with 9 terms.

Since we will trace out the photon space, we recall the partial trace formula:
\begin{IEEEeqnarray*}{rCl}
\Tr_B (\ket{a_k}\bra{a_i}\otimes \ket{b_l}\bra{b_j})  &= & \ket{a_k}\bra{a_i}\Tr(\ket{b_l}\bra{b_j}) \\
&= & \ket{a_k}\bra{a_i}\braket{b_l | b_j} \\
\end{IEEEeqnarray*}

Since $\{\ket{0},\ket{1}\}$ is an orthonormal basis of the state space $A$ of the photon, there are only 5 out of 9 terms left after the partial trace operation over the photon (those where the bit for the photon is the same in the ket and in the bra):

\begin{IEEEeqnarray*}{rCl}
\Tr_A (\ket{\varphi}\bra{\varphi}) &= & (\abs{a}^2+\abs{b}^2\sin^2(\Omega t))\ket{0}\bra{0}  + ab^*\cos(\Omega t)\ket{0}\bra{1} \\
&& \qquad +\> a^*b\cos(\Omega t)\ket{1}\bra{0} +\abs{b}^2\cos^2(\Omega t)\\
 &= &  \begin{bmatrix}
	\abs{a}^2 + (1-\abs{a}^2)\gamma & ab^*\sqrt{1-\gamma}   \\[1em]
	a^*b\sqrt{1-\gamma} & \abs{b}^2(1-\gamma)
\end{bmatrix} 
\end{IEEEeqnarray*}
with $\gamma=\sin^2(\Omega t)$. Now compare with equation~\ref{eq:5} and recall that
\begin{IEEEeqnarray*}{rCl}
\rho & =&\begin{bmatrix}\abs{a}^2 & ab^*\\[1em]
a^*b & \abs{b}^2\end{bmatrix}  
\end{IEEEeqnarray*}
to see that this is indeed the amplitude damping operation.

\section{} We consider the density operator
\begin{IEEEeqnarray*}{rCl}
\rho &= &\begin{bmatrix}p & 0\\
   0 & 1-p
\end{bmatrix}  \\
\end{IEEEeqnarray*}
 The qubit is in state $\ket{0}$ with probability $p_0=p$ and in state $\ket{1}$ with probability $p_1=1-p$.

 Let us compute $T$ as a function of $E_0$, $E_1$ and $p$:
\begin{IEEEeqnarray*}{rrCl}
&\mathcal{Z} =  \frac{e^{-\frac{E_0}{k_BT}}}{p}& = & e^{-\frac{E_0}{k_BT}}+e^{-\frac{E_1}{k_BT}} \\
\Leftrightarrow&\frac{1}{p} & = & 1 + e^{-\frac{E_1-E_0}{k_BT}} \\
\Leftrightarrow&\frac{1}{p}-1 & = &  e^{-\frac{E_1-E_0}{k_BT}} \\
\Leftrightarrow&\frac{1}{p}-1 & = &  e^{-\frac{E_1-E_0}{k_BT}} \\
\Leftrightarrow& -\frac{E_1-E_0}{k_BT} & = &  \ln(\frac{1-p}{p}) \\
\Leftrightarrow& T & = & -\frac{1}{k_B} \frac{E_1-E_0}{\ln(\frac{1-p}{p}) }\\
\end{IEEEeqnarray*}

Assuming $E_1>E_0$, \begin{itemize}
	\renewcommand\labelitemi{--}
	\item the regular amplitude damping case corresponds to $T\to 0^+$, $p=1$.
	\item When $T\to +\infty$, $p \to \frac{1}{2} $.
\end{itemize}


\clearpage
\appendix
\section{Derivation of the formula of unitary evolution for atom photon interaction}
\label{unitary}
We consider a system formed by a two-level atom and a cavity confined electric field. The Jaynes-Cummings Hamiltonian is
\begin{IEEEeqnarray*}{rCl}
H &= & \delta Z + g(a\sigma_-+a^\dagger\sigma_+)\\
\end{IEEEeqnarray*}
where $g$ is some constant which describes the strength of the interaction, $\delta=\frac{\omega-\omega_0}{2}$ is the {\em detuning}, $a^\dagger,a$ are respectively the creation,annihilation operators~\footnote{It seems to me the book mixes up $a^\dagger$ with $a$ in several places.} on the single mode field, and $\sigma_\pm$ are operators acting on the two-level atom, namely:
\begin{IEEEeqnarray*}{rCl}
\sigma_+ & =& \frac{1}{2}(X+\iu Y) \\
& = &\begin{bmatrix}
       0 & 1\\
       0 & 0
    \end{bmatrix}\\
\sigma_- & =& \frac{1}{2}(X-\iu Y) \\
& = &\begin{bmatrix}
    0 & 0\\
    1 & 0
 \end{bmatrix}\\
\end{IEEEeqnarray*}

We recall
\begin{IEEEeqnarray*}{rCl}
\forall n \in\mathbb{N},\quad a^\dagger\ket{n} &= &\sqrt{n+1}\ket{n+1}  \\
a\ket{n+1} &= &\sqrt{n+1}\ket{n}  \\
a\ket{0} & = &0
\end{IEEEeqnarray*}

The first label corresponding to electric field, the second to atom, we have:
\begin{IEEEeqnarray*}{rCl}
   Z\ket{00} &= & \ket{00} \\
   Z\ket{01} &= & -\ket{01} \\
   Z\ket{10} &= & \ket{10} \\
   a\sigma_-\ket{00} &= & 0 \\
   a\sigma_-\ket{01} &= & 0 \\
   a\sigma_-\ket{10} &= & \ket{01} \\
   a^\dagger\sigma_+\ket{00} &= & 0 \\
   a^\dagger\sigma_+\ket{01} &= & \ket{10} \\
   a^\dagger\sigma_+\ket{10} &= & 0 \\
\end{IEEEeqnarray*}
This shows that $F=\spn\{\ket{00},\ket{01},\ket{10}\}$ is an invariant subspace for $H$, i.e. $H(F)\subset F$. The same is true for $H^n, n\in\mathbb{N}$ and $U=e^{-\iu H\Delta t}=\sum \frac{(-\iu\Delta t)^n}{n!}H^n$. 

Let's find the representation of $H$ in the basis $(\ket{00}, \ket{01},\ket{10},\ket{11})$. 

The representation of $Z=I\otimes Z$ is
\begin{IEEEeqnarray*}{rCl}
I\otimes Z &= &\begin{bNiceMatrix} \Block[c]{2-2}<\LARGE>{1\boldsymbol{Z}}  & & \Block[c]{2-2}<\LARGE>{0\boldsymbol{Z}}&\\
   \hphantom{AA}&\hphantom{AA}&\hphantom{AA}&\hphantom{AA}\\
   \Block[c]{2-2}<\LARGE>{0\boldsymbol{Z}}  & & \Block[c]{2-2}<\LARGE>{1\boldsymbol{Z}}&\\
   &&&\\
\end{bNiceMatrix}  \\
&= &\begin{bNiceMatrix} \Block[c]{2-2}<\LARGE>{\boldsymbol{Z}}  & & \Block[c]{2-2}<\LARGE>{\boldsymbol{0}}&\\
   \hphantom{AA}&\hphantom{AA}&\hphantom{AA}&\hphantom{AA}\\
   \Block[c]{2-2}<\LARGE>{\boldsymbol{0}}  & & \Block[c]{2-2}<\LARGE>{\boldsymbol{Z}}&\\
   &&&\\
\end{bNiceMatrix}  \\
&= &\begin{bNiceMatrix} 1 & 0& 0&0\\
   \text{\kern 0.37em}0\text{\kern 0.37em}&\text{\kern 0.37em}-1\text{\kern 0.37em}&\text{\kern 0.37em}0\text{\kern 0.37em}&\text{\kern 0.37em}0\text{\kern 0.37em}\\
   0  &0 & 1&0\\
   0&0&0&-1\\
\end{bNiceMatrix}  \\
\end{IEEEeqnarray*}

The representation of annihilation operator in the $(\ket{n})_{n\in\mathbb{N}}$ basis of the electric field state space is
\setcounter{MaxMatrixCols}{20}
\NiceMatrixOptions{cell-space-limits = 1pt}
\pgfdeclarelayer{bg}
\pgfsetlayers{bg,main}
\begin{IEEEeqnarray*}{rCl}
a 
   & = &\begin{bNiceMatrix}[nullify-dots,first-row,last-col]
	&                                            &       & &    &        &       & & n&  &&&&&&\\
	 0&1&  & &         &             & & &  &       &&\\
	           &           & \sqrt{2}   &    &          &        &       &\Block[c]{3-3}<\LARGE>{\boldsymbol{0}}  & &  &&&&&\\
          &                                      &       &\sqrt{3} &    &        &       & & &&&&&&\\
          &                                            &       & & \sqrt{4}  &        &       & & & && &&&\\
               &                                    &       &&&             & & & &        &&&&&&\\
               &                                            & & &         &  &  &       && &&&&&\\
         &                                            &       & &    &             &                 &       &&&&&&&\\
          &                                            &       & &    &        &                                                &       &\sqrt{n} & &&&&&&n-1\\
          &                                            &       & &    &        &                                                      &       &&&&&&&\\
		  &                                            &       & &    &        &       & & &  &&&&&\\
		  &                                            & \Block[c]{3-3}<\LARGE>{\boldsymbol{0}}       & &    &        &       & & &  &&&&&\\
		  &                                            &       & &    &        &       & & &  &&&&&\\
		  &                                            &       & &    &        &       & & &  &&&&&\\
		  &                                            &       & &    &        &       & & &  &&&&&\NotEmpty\\
               &                                      & & &         &             &             &&                                   &       &&&&&\NotEmpty
   \CodeAfter
   \begin{tikzpicture}[]
	
	%\draw (1,1) -- (2,1)--(0,0);
   %\draw (2-2) circle (2mm);
   %\draw (5-5) circle (2mm);
   %\draw (2-2) -- (5-5);
   %\draw (-5,-5) -- (5,5);
   %\draw[dot diameter=1pt, dot spacing=5pt, dots] (-7,1.97) -- (-5.1,1.27);
   \draw[dot diameter=1pt, dot spacing=5pt, dots] (-3.5,-0.1) -- (-1.1,-2.2);
   
   \draw[dot diameter=1pt, dot spacing=5pt, dots] (1-1.south) -- (15-15.west);
   \draw[dot diameter=1pt, dot spacing=5pt, dots] (4-5.south) -- (8-9.west);
   %\draw[dot diameter=1pt, dot spacing=5pt, dots] (3-10.south) -- (5-12.west);
   %\draw[dot diameter=1pt, dot spacing=5pt, dots] (5-12.south) -- (7-14.west);
   %\draw[dot diameter=1pt, dot spacing=5pt, dots] (1-2.south) -- (1-5.south west);
   %\draw[dot diameter=1pt, dot spacing=5pt, dots] (6-7.south) -- (6-10.south west);
   %\line{2-2}{5-5}.north east;
   %\line{2-2}{1-5}.north;

   \begin{pgfonlayer}{bg}
	%\draw (1-8) -- (8-15); 
   \end{pgfonlayer}
   \end{tikzpicture}\end{bNiceMatrix}
\end{IEEEeqnarray*}

The representation of $a\otimes\sigma_-$ is then
\begin{IEEEeqnarray*}{rCl}
   a\otimes\sigma_- &= &\begin{bNiceMatrix} \Block[c]{2-2}<\LARGE>{0\boldsymbol{\sigma_-}}  & & \Block[c]{2-2}<\LARGE>{1\boldsymbol{\sigma_-}}&\\
      \hphantom{AA}&\hphantom{AA}&\hphantom{AA}&\hphantom{AA}\\
      \Block[c]{2-2}<\LARGE>{0\boldsymbol{\sigma_-}}  & & \Block[c]{2-2}<\LARGE>{0\boldsymbol{\sigma_-}}&\\
      &&&\\
   \end{bNiceMatrix}  \\
   &= &\begin{bNiceMatrix} \Block[c]{2-2}<\LARGE>{\boldsymbol{0}}  & & \Block[c]{2-2}<\LARGE>{\boldsymbol{\sigma_-}}&\\
      \hphantom{AA}&\hphantom{AA}&\hphantom{AA}&\hphantom{AA}\\
      \Block[c]{2-2}<\LARGE>{\boldsymbol{0}}  & & \Block[c]{2-2}<\LARGE>{\boldsymbol{0}}&\\
      &&&\\
   \end{bNiceMatrix}  \\
   &= &\begin{bNiceMatrix} 0 & 0& 0&0\\
      \text{\kern 0.4em}0\text{\kern 0.4em}&\text{\kern 0.4em}0\text{\kern 0.4em}&\text{\kern 0.4em}1\text{\kern 0.4em}&\text{\kern 0.4em}0\text{\kern 0.4em}\\
      0  &0 & 0&0\\
      0&0&0&0\\
   \end{bNiceMatrix}  \\
   \end{IEEEeqnarray*}

Similarly,
\begin{IEEEeqnarray*}{rCl}
   a^\dagger\otimes\sigma_+ &= &\begin{bNiceMatrix} \Block[c]{2-2}<\LARGE>{0\boldsymbol{\sigma_+}}  & & \Block[c]{2-2}<\LARGE>{0\boldsymbol{\sigma_-}}&\\
      \hphantom{AA}&\hphantom{AA}&\hphantom{AA}&\hphantom{AA}\\
      \Block[c]{2-2}<\LARGE>{1\boldsymbol{\sigma_+}}  & & \Block[c]{2-2}<\LARGE>{0\boldsymbol{\sigma_+}}&\\
      &&&\\
   \end{bNiceMatrix}  \\
   &= &\begin{bNiceMatrix} \Block[c]{2-2}<\LARGE>{\boldsymbol{0}}  & & \Block[c]{2-2}<\LARGE>{\boldsymbol{0}}&\\
      \hphantom{AA}&\hphantom{AA}&\hphantom{AA}&\hphantom{AA}\\
      \Block[c]{2-2}<\LARGE>{\boldsymbol{\sigma_+}}  & & \Block[c]{2-2}<\LARGE>{\boldsymbol{0}}&\\
      &&&\\
   \end{bNiceMatrix}  \\
   &= &\begin{bNiceMatrix} 0 & 0& 0&0\\
      \text{\kern 0.4em}0\text{\kern 0.4em}&\text{\kern 0.4em}0\text{\kern 0.4em}&\text{\kern 0.4em}0\text{\kern 0.4em}&\text{\kern 0.4em}0\text{\kern 0.4em}\\
      0  &1 & 0&0\\
      0&0&0&0\\
   \end{bNiceMatrix}  \\
   \end{IEEEeqnarray*}

  Thus the representation of $H$ in the basis $(\ket{00}, \ket{01},\ket{10},\ket{11})$ is 
  \begin{IEEEeqnarray*}{rCl}
   H &= &\begin{bNiceMatrix} \delta & 0& 0&0\\
      \text{\kern 0.37em}0\text{\kern 0.37em}&\text{\kern 0.37em}-\delta\text{\kern 0.37em}&\text{\kern 0.37em}g\text{\kern 0.37em}&\text{\kern 0.37em}0\text{\kern 0.37em}\\
      0  &g & \delta&0\\
      0&0&0&-\delta\\
   \end{bNiceMatrix}  \\
   \end{IEEEeqnarray*}

   The representation of the restriction of $H$ in the basis $(\ket{00}, \ket{01},\ket{10})$ is 
   \begin{IEEEeqnarray*}{rCl}
    H &= &\begin{bNiceMatrix} \text{\kern 0.37em}\delta\text{\kern 0.37em}&  \text{\kern 0.37em}0\text{\kern 0.37em}&  \text{\kern 0.37em}0\text{\kern 0.37em}\\[0.8em]
       0&\text{\kern 0.37em}-\delta\text{\kern 0.37em}&\text{\kern 0.37em}g\text{\kern 0.37em}\\[0.8em]
       0  &g & \delta\\
    \end{bNiceMatrix}  \\
    &= &\begin{bNiceMatrix} \text{\kern 0.37em}\delta\text{\kern 0.37em}&  \text{\kern 0.37em}0\text{\kern 0.37em}&  \text{\kern 0.37em}0\text{\kern 0.37em}\\[0.8em]
      \text{\kern 0.37em}0\text{\kern 0.37em}&\Block[c]{2-2}<\LARGE>{\boldsymbol{H_1}} &\\[0.8em]
      0  & & \\
   \end{bNiceMatrix}  \\
    \end{IEEEeqnarray*}

Block calculus shows that
\begin{IEEEeqnarray*}{rCl}
e^{-\iu H \Delta t} &= & \begin{bNiceMatrix} \text{\kern 0.37em}e^{-\iu \delta \Delta t}\text{\kern 0.37em}&  \text{\kern 0.37em}0\text{\kern 0.37em}&  \text{\kern 0.37em}0\text{\kern 0.37em}\\[0.8em]
   \text{\kern 0.37em}0\text{\kern 0.37em}&\Block[c]{2-2}<\large>{\boldsymbol{e^{-\iu H_1\Delta t}}} &\\[0.8em]
   0  & & \\
\end{bNiceMatrix}  \\
\end{IEEEeqnarray*}

Let $\Omega=\sqrt{g^2+\delta^2}$, the {\em Rabi frequency}.

\begin{IEEEeqnarray*}{rCl}
H_1^2 &= &\begin{bmatrix} \Omega^2 & 0\\[0.8em]
   0 & \Omega ^2
\end{bmatrix}  \\
&= & \Omega^2 I_2 \\
\end{IEEEeqnarray*}

This shows that
\begin{IEEEeqnarray*}{rCl}
\forall n\in\mathbb{N},\quad H_1^{2n} &= & \Omega^{2n} I_2  \\
H_1^{2n+1} &= & \Omega^{2n} H_1  \\
\end{IEEEeqnarray*}

Then,
\begin{IEEEeqnarray*}{rCl}
e^{-\iu H_1\Delta t} &= &\sum_{n=0}^{+\infty}\frac{(-\iu \Omega\Delta t)^{2n}}{(2n)!}I_2+\frac{1}{\Omega}\sum_{n=0}^{+\infty}\frac{(-\iu \Omega\Delta t)^{2n+1}}{(2n+1)!}H_1  \\
&= &\sum_{n=0}^{+\infty}(-1)^n\frac{(\Omega\Delta t)^{2n}}{(2n)!}I_2-\iu\frac{1}{\Omega}\sum_{n=0}^{+\infty}(-1)^n\frac{( \Omega\Delta t)^{2n+1}}{(2n+1)!}H_1  \\
& = &\cos(\Omega t)I_2 -\iu \frac{1}{\Omega} \sin(\Omega t)H_1
\end{IEEEeqnarray*}

Finally the matrix $U$ is 
\begin{IEEEeqnarray*}{rCl}
e^{-\iu H\Delta t} &= & \begin{bNiceMatrix}[nullify-dots,first-row,last-col]
   \text{\tiny $U\ket{00}$}&\text{\tiny $U\ket{01}$}&\text{\tiny $U\ket{10}$}\\
   \text{\kern 0.37em}e^{-\iu \delta \Delta t}\text{\kern 0.37em}&  \text{\kern 0.37em}0\text{\kern 0.37em}&  \text{\kern 0.37em}0\text{\kern 0.37em}&\text{\tiny $\ket{00}$}\\[0.8em]
   \text{\kern 0.37em}0\text{\kern 0.37em}&\cos(\Omega t)+\iu \frac{\delta}{\Omega}\sin(\Omega t) &-\iu \frac{g}{\Omega}\sin(\Omega t)&\text{\tiny$\ket{01}$}\\[0.8em]
   0  &-\iu \frac{g}{\Omega}\sin(\Omega t) &\cos(\Omega t)-\iu \frac{\delta}{\Omega}\sin(\Omega t)&\text{\tiny$\ket{10}$}\\
\end{bNiceMatrix}  \\
\end{IEEEeqnarray*}

\section{Circuit model for amplitude damping}

We want to prove that the following circuit models the amplitude damping operation

\begin{center}
\begin{quantikz}
\lstick{$\alpha\ket{0}+\beta\ket{1}$}    &  \ctrl{1}           &  \targ{}  &   \qw   & \qw \\
\lstick{$\ket{0}$} & \gate{R_y(\theta)}  & \ctrl{-1} &  \meter{} & \qw
\end{quantikz}\end{center}

Recall that
\begin{IEEEeqnarray*}{rCl}
R_y(\theta) &= & e^{-\iu \frac{\theta}{2}Y} \\
& = &\begin{bmatrix}
					\cos(\frac{\theta}{2}) & -\sin(\frac{\theta}{2})  \\[1em]
					\sin(\frac{\theta}{2}) &\cos(\frac{\theta}{2})
\end{bmatrix}
\end{IEEEeqnarray*}

Initially the two-qubit state is
\begin{IEEEeqnarray*}{rCl}
 (\alpha\ket{0}+\beta\ket{1})\ket{0}&= & \alpha\ket{00}+\beta\ket{10}  \\
\end{IEEEeqnarray*}

After the controlled $R_y$ gate it becomes
\begin{IEEEeqnarray*}{rCl}
	\alpha\ket{00}+\beta\ket{1}R_y(\theta)\ket{0} 	& =&  \alpha\ket{00}+\beta\ket{1}(\cos(\frac{\theta}{2})\ket{0}+\sin(\frac{\theta}{2})\ket{1})  \\
	& =&  \alpha\ket{00}+\beta(\cos(\frac{\theta}{2})\ket{10}+\sin(\frac{\theta}{2})\ket{11})  \\
\end{IEEEeqnarray*}

After the controlled not gate,
\begin{IEEEeqnarray*}{rCl}
 & & \alpha\ket{00}+\beta(\cos(\frac{\theta}{2})\ket{10}+\sin(\frac{\theta}{2})\ket{01})  \\
\end{IEEEeqnarray*}

This is the effect of amplitude damping, with probability of 1 be switched to 0, or one photon being lost to environment, being $\gamma=\sin^2(\frac{\theta}{2})$.

\end{document}
